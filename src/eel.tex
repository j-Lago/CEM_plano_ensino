%! Author = j-Lago
%! Date = 30-Mar-22
%! suppress = Unicode

\begin{table}[!hp]
\begin{tabular}{c|ccc|}
\multicolumn{1}{l|}{}   & \multicolumn{3}{c|}{2022.2} \\
\multicolumn{1}{l|}{}   & \multicolumn{1}{c|}{\rotatebox{90}{qui}} & \multicolumn{1}{c|}{\rotatebox{90}{sex}} & \multicolumn{1}{c|}{\rotatebox{90}{sáb}} \\ \hhline{-===}
\multirow{4}{*}{\rotatebox[origin=c]{90}{ago}}\
                        & \feriado{4}             & \feriado{5}             &                         \\
                        & \feriado{11}            & \feriado{12}            &                         \\
                        & 18                      & 19                      & \anp{20}                \\
                        & 25                      & 26                      &                         \\ \cline{1-4}
\multirow{5}{*}{\rotatebox[origin=c]{90}{set}}\
                        & 1                       & \pratica{2}             &                         \\
                        & 8                       & \pratica{9}             &                         \\
                        & 15                      & 16                      &                         \\
                        & \prova{22}              & \pratica{23}            &                         \\
                        & 29                      & \pratica{30}            &                         \\ \cline{1-4}
\multirow{4}{*}{\rotatebox[origin=c]{90}{out}}\
                        & \prova{6}               & \pratica{7}             &                         \\
                        & 13                      & 14                      &                         \\
                        & 20                      & \pratica{21}            &                         \\
                        & 27                      & \feriado{28}            &                         \\ \cline{1-4}
\multirow{4}{*}{\rotatebox[origin=c]{90}{nov}}\
                        & 3                       & \pratica{4}             & \extra{5}               \\
                        & 10                      & \pratica{11}            &                         \\
                        & 17                      & \pratica{18}            &                         \\
                        & 24                      & \feriado{25}            &                         \\ \cline{1-4}
\multirow{4}{*}{\rotatebox[origin=c]{90}{dez}}\
                        & \prova{1}                & \pratica{2}            &                          \\
                        & 8                        & \seminario{9}          &                          \\
                        & \prova{15}               & \seminario{16}         &                          \\
                        & 22                       &                        &                          \\ \cline{1-4}
\multicolumn{4}{l}{\small\ } \\
\multicolumn{4}{l}{\small\feriado{-- feriados}} \\
\multicolumn{4}{l}{-- teóricas}\\
\multicolumn{4}{l}{\pratica{-- práticas}}\\
\multicolumn{4}{l}{\prova{-- avaliações}}\\
\multicolumn{4}{l}{\anp{-- ANPs sáb.}}\\
\multicolumn{4}{l}{\extra{-- ANPs extras}}\\
\multicolumn{4}{l}{\seminario{-- seminário}}
\end{tabular}
\quad \hfill
\begin{tabular}{c|cl}
    \multirow{5}{*}{\rotatebox[origin=c]{90}{agosto}}
    & 18                        & apresentação do plano de ensino; ex00\\
    & 19                        & revisão de circuitos;\\
    & \anp{20}                  & \anp{revisão de eletromagnetismo}\\
    & 25                        & transformador ideal; reflexão de impedância; isolação; ex01 \\
    & 26                        & não idealidade e circuito equivalente; ex02 \\\hline
    \multirow{10}{*}{\rotatebox[origin=c]{90}{setembro}}\
    & 1                         & circuito equivalente; rendimento; regulação; ex03 \\
    & \pratica{2}               & \pratica{ensaios de magnetização; relação de transformação; polaridade} \\
    & 8                         & estimação de parâmetros; autotransformador; ex04\\
    & \pratica{9}               & \pratica{ensaios em vazio e de curto--circuito do transformador mono; ex04} \\
    & 15                        & transformador trifásico; ex05 \\
    & 16                        & dúvidas \\
    & \prova{22}                & \prova{P1: prova sobre transformadores} \\
    & \pratica{23}              & \pratica{ensaios do transformador monofásico com carga; ex04} \\
    & 29                        & correção da prova P1 \\
    & \pratica{30}              & \pratica{ensaios do transformador trifásico} \\ \hline
    \multirow{8}{*}{\rotatebox[origin=c]{90}{outubro}}\
    & \prova{6}                 & \prova{R1: recuperação da prova sobre transformadores}\\
    & \pratica{7}               & \pratica{continuação dos ensaios do transformador trifásico} \\
    & 13                        & introdução aos motores elétricos \\
    & 14                        & circuito do estator e campo girante; ex06\\
    & 20                        & circuito do rotor e torque induzido \\
    & \pratica{21}              & \pratica{aula para finalizar/refazer ensaios não concluídos} \\
    & 27                        & circuito equivalente monofásico do MIT; ex07\\
    & \feriado{28}              & \feriado{feriado}\\ \hline
    \multirow{9}{*}{\rotatebox[origin=c]{90}{novembro}}\
    & 3                         & circuito equivalente; torque partida; torque máximo; ex08\\
    & \pratica{4}               & \pratica{aspectos construtivos, dados de placa, partida e reversão do motor}\\
    & \extra{5}                 & \extra{trabalho de pesquisa sobre motores especiais}\\
    & 10                        & velocidade de acomodação; gerador de indução; ex08\\
    & \pratica{11}              & \pratica{ensaio com rotor travado e livre}\\
    & 17                        & motor de indução monofásico\\
    & \pratica{18}              & \pratica{levantamento dos parâmetros do MIT; ex09}\\
    & 24                        & dúvidas \\
    & \feriado{25}              & \feriado{feriado}\\ \hline
    \multirow{7}{*}{\rotatebox[origin=c]{90}{dezembro}}\
    & \prova{1}                 & \prova{P2: prova sobre motores de indução}\\
    & \pratica{2}               & \pratica{ensaio do MIT com carga}\\
    & 8                         & correção prova P2\\
    & \seminario{9}             & \seminario{apresentação dos trabalhos sobre motores}\\
    & \prova{15}                & \prova{R2: recuperação da prova sobre motores de indução}\\
    & \seminario{16}            & \seminario{apresentação dos trabalhos sobre motores}\\
    & 22                        & divulgação dos resultados e encerramento
\label{tab:aulas_eel}
\end{tabular}
\end{table}
