%! Author = j-Lago
%! Date = 29-Mar-22
%! suppress = Unicode

% preamble
\documentclass[10pt, a4paper]{article}
\usepackage{../modelo/plano}
\usepackage{wrapfig}
%\usepackage{pythontex}
\usepackage[normalem]{ulem}
\usepackage{hhline}
\usepackage{etoolbox}

\ifx\conditionmacro\undefined
    \immediate\write18{pdflatex -jobname= "eeln"  -output-directory="../out/" "\gdef\string\conditionmacro{eeln}\string\input\space\jobname"}
    \immediate\write18{pdflatex -jobname= "eel"   -output-directory="../out/" "\gdef\string\conditionmacro{eel}\string\input\space\jobname"}
\expandafter\stop
\fi


\def\eeln{eeln}


\ifx\conditionmacro\eeln
    \subject{(CEE22106) Conversão Eletromecânica da Energia I}
    \course{Engenharia Eletrônica}
\else
    \subject{(CEM22005) Conversão Eletromecânica de Energia I}
    \course{Engenharia Elétrica}
\fi

\author{Jackson Lago, Dr. Eng.}
\title{Plano de Ensino}


\subtitle{\thesubject}
\shorttitle{\thetitle}
\date{\today}

% document
\begin{document}

    \capa

    %\onehalfspacing
    \setstretch{1.2}


    \subsection*{Metodologia}

    Aulas expositivas, dialogadas e práticas.

    \hspace{3mm}


    $\dagger$ \emph{Excepcionalmente nesse semestre, devido às restrições decorrentes da pandemia do COVID--19 e para complementação do curto calendário
    acadêmico 2022.2, parte das aulas serão ministradas na modalidade não presencial (ANP), em acordo com a regulamentação vigente no IFSC
    campus Florianópolis.}

    \subsection*{Ementa}
    \begin{enumerate}[label=\roman*)]
        \item Transformadores monofásicos, trifásicos e autotransformadores;
        \item Motores de indução trifásicos e monofásicos;
        \item Motores especiais: motor universal, motor com espira de sombra, motor a imã permanente e motor de passo.
    \end{enumerate}


    \subsection*{Competências}

    Compreender o funcionamento de máquinas elétricas a partir dos fenômenos eletromagnéticos, da análise de seus circuitos equivalentes e
    de ensaios práticos.


    \subsection*{Habilidades}
    \begin{enumerate}[label=\roman*)]
        \item Analisar e descrever os elementos construtivos básicos dos transformadores, motores de indução e motores
especiais;
        \item Analisar e descrever os fenômenos eletromagnéticos nos quais se baseiam o funcionamento dos transformadores,
motores de indução e motores especiais;
        \item Analisar e descrever as características operativas dos transformadores, motores de indução e motores especiais,
para diferentes condições de operação;
        \item Calcular os valores das grandezas características do funcionamento de transformadores, motores de indução e
motores especiais, utilizando os respectivos circuitos equivalentes;
        \item Realizar ensaios e outras observações práticas visando medir e calcular os valores das grandezas características do
funcionamento de transformadores, motores de indução e motores especiais.
    \end{enumerate}

    \subsection*{Requisitos}
    \begin{itemize}
        \item Eletromagnetismo I
        \item Circuitos Elétricos II
    \end{itemize}


    \subsection*{Avaliação da Aprendizagem}
    A avaliação da aprendizagem será feita através de duas provas individuais e sem consulta ($P_1$ e $P_2$), e por um trabalho de
    pesquisa ($T$) a ser apresentado na forma de seminário.
    Os valores de referência para os pesos dessas avaliações e seus conteúdos são:
    \begin{itemize}[itemsep=0mm]
    \item[$P_1$:] (peso 40\%) transformadores elétricos
    \item[$P_2$:] (peso 50\%) motores elétricos de indução
    \item[$T$:] (peso 10\%) motores especiais
    \end{itemize}

    Segundas chamadas serão realizadas \textbf{exclusivamente} para os casos previstos no Art. 162 da RDP e deverão ser solicitadas diretamente à coordenação
    do curso (e não ao professor que leciona a disciplina).

    As avaliações serão individuais.
    Respostas iguais de dois alunos ou \textbf{cópias} de livros ou internet, \textbf{mesmo que parciais} por fragmentos de textos ou figuras, \textbf{configuram plágio} e,
    como consequência, será atribuída nota geral zero para a avaliação (de ambos os alunos).

    Resultados numéricos errados, mesmo que com procedimentos parcialmente corretos, não pontuarão nas avaliações.

    Resultados numéricos corretos, mas sem desenvolvimento de cálculo consistente não pontuarão nas avaliações.

    Como previsto pelo regulamento didático--pedagógico do IFSC, para as duas provas ($P_1$ e $P_2$) serão realizadas recuperações ($R_1$ e $R_2$).
    A nota a ser registrada será o maior valor entre a nota da prova e a de sua recuperação.

    As notas de cada avaliação serão registradas em valores inteiros de 0 a 10, adotando critério de arredondamento da ABNT quando necessário.

    A nota final será calculada pela média ponderada das avaliações:
    %\vspace{-1mm} \[ M = 0.4 \cdot  \text{max}(P_1, R_1) + 0.5 \cdot  \text{max}(P_2, R_2) +  0.1 \cdot T \] \vspace{-1mm}
    \[ M = \dfrac{4 \cdot  \max(P_1, R_1) + 5 \cdot  \max(P_2, R_2) +  T}{10} \]
    e será registrada em valores inteiros de 0 a 10.
    Valores para a média ponderada final iguais ou superiores a 5.5 poderão ser arredondados tanto para 5 como para 6, dependendo da participação e
    desempenho do aluno nas atividades práticas de laboratório e demais atividades propostas.
    Valores da média ponderada final inferiores a 5.5 serão arredondados para 5.

    Será aprovado o aluno que atingir nota final \textbf{superior} ou \textbf{igual} a 6.

    Adicionalmente, para aprovação, o aluno deverá acumular presença \textbf{superior} ou \textbf{igual} a 75\%.
    Caso esse percentual mínimo não for atingido, o aluno estará \textbf{reprovado} e a ele será atribuída a nota \textbf{zero}.
    Recomenda--se ao aluno o acompanhamento semanal dos registros de frequência pelo SIGAA.

    %\nocite{*}
    %\bibliography{main}
    %\bibliographystyle{unsrt}

    \subsection*{Atendimento Extraclasse}
    Terças--feiras as 12:30 e quintas--feiras as 17:30 na sala G102b.

    \subsection*{Bibliografia Básica}
    \begin{enumerate}[label={[}\arabic*{]}, leftmargin=*, itemsep=-0mm]
    \item Fitzgerald, A. E.; Kingsley, Charles Jr.; Umans, Stephen D. \emph{Máquinas Elétricas}. McGraw–Hill, $6^{\underline{\text{a}}}$ edição, 2006.
    \item Chapman, Stephen. \emph{Fundamentos de Máquinas Elétricas}. McGraw–Hill, $5^{\underline{\text{a}}}$ edição, 2013.
    \end{enumerate}

    \subsection*{Bibliografia Complementar}
    \begin{enumerate}[label={[}\arabic*{]}, resume, leftmargin=*, itemsep=0mm]
    \item Kosow, Irwing L. \emph{Máquinas Elétricas e Transformadores}. Globo, $15^{\underline{\text{a}}}$ edição, 1996.
    \item Toro, Vicent del. \emph{Fundamentos de Máquinas Elétricas}. LTC, $6^{\underline{\text{a}}}$ edição, 2006.
    \item Wasynczuk, Oleg; Krause, Paul C.; Sudhoff, Scott D.; Pekarek, Steven. \emph{Analysis of Electrical
Machinery and Drive Systems}. Willey, 2002.
    \end{enumerate}

    \subsection*{Calendário e Planejamento}

    \newcommand{\feriado}[1]{\color{red!80}\sout{#1}}
    \newcommand{\anp}[1]{\color{RoyalBlue}#1}
    \newcommand{\pratica}[1]{\color{teal}#1}
    \newcommand{\prova}[1]{\color{Purple}#1}
    \newcommand{\extra}[1]{\color{Bittersweet}#1}
    \newcommand{\seminario}[1]{\color{Rhodamine}#1}

    \input{\conditionmacro}


\end{document}