%! Author = j-Lago
%! Date = 30-Mar-22
%! suppress = Unicode

\begin{table}[!hp]
\begin{tabular}{c|ccc|}
\multicolumn{1}{l|}{}   & \multicolumn{3}{c|}{2022.2} \\
\multicolumn{1}{l|}{}   & \multicolumn{1}{c|}{\rotatebox{90}{ter}} & \multicolumn{1}{c|}{\rotatebox{90}{qui}} & \multicolumn{1}{c|}{\rotatebox{90}{sáb}} \\ \hhline{-===}
\multirow{5}{*}{\rotatebox[origin=c]{90}{ago}}\
                        & \feriado{2}           & \feriado{4}                         &                         \\
                        & \feriado{9}           & \feriado{11}                        &                         \\
                        & \feriado{16}          & 18                                  & \anp{20}                \\
                        & 23                    & 25                                  &                         \\ \cline{3-4}
\multirow{6}{*}{\rotatebox[origin=c]{90}{set}}\
                        & 30                    & \multicolumn{1}{|c}{1}           &                  \\ \cline{1-2}
                        & \pratica{6}           & 8                                   &                         \\
                        & \pratica{13}          & 15                                  &                         \\
                        & 20                    & \prova{22}                          &                         \\
                        & \pratica{27}          & 29                                  &                         \\ \cline{1-4}
\multirow{4}{*}{\rotatebox[origin=c]{90}{out}}\
                        & \pratica{4}           & \prova{6}                           &                         \\
                        & \pratica{11}          & 13                                  &                         \\
                        & 18                    & 20                                  &                         \\
                        & \pratica{25}          & 27                                  &                         \\ \cline{1-4}
\multirow{5}{*}{\rotatebox[origin=c]{90}{nov}}\
                        & \pratica{1}           & 3                                   & \extra{5}               \\
                        & \pratica{8}           & 10                                  &                         \\
                        & \feriado{15}          & 17                                  &                         \\
                        & \pratica{22}          & 24                                  &                         \\ \cline{3-4}
\multirow{5}{*}{\rotatebox[origin=c]{90}{dez}}\
                        & \pratica{29}          & \multicolumn{1}{|c}{\prova{1}}              &                  \\ \cline{1-2}
                        & \seminario{6}         & 8                                   &                          \\
                        & \seminario{13}        & \prova{15}                          &                          \\
                        & 20                    & 22                                  &                          \\ \cline{1-4}
\multicolumn{4}{l}{\small\ } \\
\multicolumn{4}{l}{\small\feriado{-- feriados}} \\
\multicolumn{4}{l}{-- teóricas}\\
\multicolumn{4}{l}{\pratica{-- práticas}}\\
\multicolumn{4}{l}{\prova{-- avaliações}}\\
\multicolumn{4}{l}{\anp{-- ANPs sáb.}}\\
\multicolumn{4}{l}{\extra{-- ANPs extras}}\\
\multicolumn{4}{l}{\seminario{-- seminário}}
\end{tabular}
\quad \hfill
\begin{tabular}{c|cl}
    \multirow{5}{*}{\rotatebox[origin=c]{90}{agosto}}
    & 18                        & apresentação do plano de ensino; ex00\\
    & \anp{20}                  & \anp{revisão de eletromagnetismo}\\
    & 23                        & revisão de circuitos;\\
    & 25                        & transformador ideal; reflexão de impedância; isolação; ex01 \\
    & 30                        & não idealidade e circuito equivalente; ex02 \\\hline
    \multirow{9}{*}{\rotatebox[origin=c]{90}{setembro}}\
    & 1                         & circuito equivalente; rendimento; regulação; ex03 \\
    & \pratica{6}               & \pratica{ensaios de magnetização; relação de transformação; polaridade} \\
    & 8                         & estimação de parâmetros; autotransformador; ex04\\
    & \pratica{13}              & \pratica{ensaios em vazio e de curto--circuito do transformador mono; ex04} \\
    & 15                        & transformador trifásico; ex05 \\
    & 20                        & dúvidas \\
    & \prova{22}                & \prova{P1: prova sobre transformadores} \\
    & \pratica{27}              & \pratica{ensaios do transformador monofásico com carga; ex04} \\
    & 29                        & correção da prova P1 \\ \hline
    \multirow{8}{*}{\rotatebox[origin=c]{90}{outubro}}\
    & \pratica{4}               & \pratica{ensaios do transformador trifásico} \\
    & \prova{6}                 & \prova{R1: recuperação da prova sobre transformadores}\\
    & \pratica{11}              & \pratica{continuação dos ensaios do transformador trifásico} \\
    & 13                        & introdução aos motores elétricos \\
    & 18                        & circuito do estator e campo girante; ex06\\
    & 20                        & circuito do rotor e torque induzido \\
    & \pratica{25}              & \pratica{aula para finalizar/refazer ensaios não concluídos} \\
    & 27                        & circuito equivalente monofásico do MIT; ex07\\ \hline
    \multirow{10}{*}{\rotatebox[origin=c]{90}{novembro}}\
    & \pratica{1}               & \pratica{aspectos construtivos, dados de placa, partida e reversão do motor} \\
    & 3                         & circuito equivalente; torque partida; torque máximo; ex08\\
    & \extra{5}                 & \extra{trabalho de pesquisa sobre motores especiais}\\
    & \pratica{8}               & \pratica{ensaio com rotor travado e livre}\\
    & 10                        & velocidade de acomodação; gerador de indução; ex08\\
    & \feriado{15}              & \feriado{feriado} \\
    & 17                        & motor de indução monofásico\\
    & \pratica{22}              & \pratica{levantamento dos parâmetros do MIT; ex09}\\
    & 24                        & dúvidas \\
    & \pratica{29}              & \pratica{ensaio do MIT com carga} \\ \hline
    \multirow{6}{*}{\rotatebox[origin=c]{90}{dezembro}}\
    & \prova{1}                 & \prova{P2: prova sobre motores de indução}\\
    & \seminario{6}             & \pratica{apresentação dos trabalhos sobre motores}\\
    & 8                         & correção prova P2\\
    & \seminario{13}             & \seminario{apresentação dos trabalhos sobre motores}\\
    & \prova{15}                & \prova{R2: recuperação da prova sobre motores de indução}\\
    & 20                        & divulgação dos resultados e encerramento
\label{tab:aulas_eeln}
\end{tabular}
\end{table}